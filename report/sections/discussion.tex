\section{Discussion}
\subsection{Main Experimental Observations}
In the process of our experiment, we found that besides the property of different optimality solving method, the way of implementation could largely influence the speed. For example, using \textit{C} or \textit{Matlab} could be usually faster than \textit{Python} especially in the case that there are multi-layers loops, while it's quite common in the methods we study.

The other important factor we found that could influence the speed is the process of taking gradient and hessian. If we use the \textbf{auto-gradient toolkit}, it would be the slowest since every time it will use the \textbf{computational graph}, which is a data structure that can largely influence the speed. For the \textbf{symbolic-gradient toolkit}, we have most of our time-cost at the process of getting the expression. After such process, the calculation speed will be much faster than that of the auto-gradient method. However, this would still cause some problems. First, it's not so convenient that when $n$ changes, we need to recaculate the symbolic expression. Second, when $n$ is very large, and consider that the hessian would be very sparse, the evaling process will be very slow. Manually get the gradient and hessian expression and define corresponding function in the program would be the fastest way, but we will then lose some flexibility while it's well specified.

One simple way that we can improve the speed in the Minimal Surface problem is to utilize the optimal surface share the same \textbf{symmetry} as that of the boundary. For example, we know that for the boundary generated by $r_3(x,y) = \frac{1}{2} - |y - \frac{1}{2}|$ and $\Omega = (0,1) \times (0,1) \in R^2$ , $\Gamma=\partial \Omega$, we will have $f(\frac{1}{2}-x)=f(\frac{1}{2}+x)$ as well as $f(\frac{1}{2}-y)=f(\frac{1}{2}+y)$, so we can reduce the number of decision variables by $\frac{3}{4}$ as long as we maintain these two equations.

Finally, we would like to say that good \textbf{isolation and encapsulation} of our codes help us to carry out experiments more efficiently. You can utilize any instance that is a subclass \texttt{Solver} and use its implemented \texttt{solve} function to solve an instance that is a subclass of \texttt{Solver}. In the implementation of the quadratic penalty method, we even let our solver hold the reference of an Armijo Gradient Method. You can check the code on \textbf{Github}\footnote{\url{https://github.com/WhiskyChoy/opt_final_project}} and initialize any problem or solver as you wish.

\subsection{Extensions and Applications}
As indicated in paper \cite{yueting2007compact}, the compact form of L-BFGS is more efficient to adapt for sparse cases. Moreover, the compact form has an analog for the direct update. The compact representation of the quasi-Newton updating matrix is derived to the use in the form of limited memory update in which the vector is replaced by a modified vector so that more available information about the function can be employed to increase the accuracy of Hessian approximations. The global convergence of the proposed method is also proved \cite{yueting2007compact}. It’s expected that the compact form of the L-BFGS can provide a new model for solving more kinds of large scale optimization. 

Besides, for the existing problems in this project, we may consider more advance and sufficient algorithms to solve the defined minimum surfaces and obstacle problems, such as applying augmented-lagrange method to modify the performance of penalty parameter and Alternating Direction Method of Multiplier (ADMM) where alternate minimization are utilized to decouple sets of variables that are coupled within the augmented Lagrangian.

We can also extand our problems to a more common case. For instance, we assume that obstacle can move freely within the target set and determine what is the maxmimum minimal area. This general problem can be solved by 
sequential quadratic programming. Considering paper \cite{caffarelli2016obstacle}, the authors also discuss the regularity of the solution and of the free boundary for certain obstacle type problems involving classical minimal surfaces and nonlocal minimal surfaces, which can be further investigated.

Furthermore, in \cite{attouch2014variational}, researchers have pointed out several application scenerios such as fluid filtration in porousmedia, constrained heating, elasto-plasticity and optimal control. Since the obstacle problem has some potential physical meaning, we may considering combining additional physical definition to solve the pracitical obstacle problems.
