\section{Discussion}
\subsection{Main Experimental Observations}
\subsection{Extensions and Applications}
For the existing problems in this project, we may consider more advance and sufficient algorithms to solve the defined minimum surfaces and obstacle problems, such as applying augmented-lagrange method to modify the performance of penalty parameter and Alternating Direction Method of Multiplier (ADMM) where alternate minimization are utilized to decouple sets of variables that are coupled within the augmented Lagrangian.

We can also extand our problems to a more common case. For instance, we assume that obstacle can move freely within the target set and determine what is the maxmimum minimal area. This general problem can be solved by 
sequential quadratic programming. In \cite{caffarelli2016obstacle}, the authors also discuss the regularity of the solution and of the free boundary for certain obstacle type problems involving classical minimal surfaces and nonlocal minimal surfaces, which can be further investigated.

In \cite{attouch2014variational}, researchers have pointed out several application scenerios such as fluid filtration in porousmedia, constrained heating, elasto-plasticity and optimal control. Since the obstacle problem has some potential pysical meaning, we may considering combining additional physical definition to solve the pracitical obstacle problems.